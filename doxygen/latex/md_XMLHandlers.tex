\subsection*{How it works}

Reads the xml file in a specific way to display it. Every xml file needs to be based on a specific validation file (xsd). It need to have the filename based on the xml schema. The xml syntax is read through a

Eg.\+: Tanach-\/xml has various book files. The folder for the xml files needs to have the same name as the schema. (tanach-\/xml). Tanach-\/xml needs a validation file in xsd format (tanach-\/xml.\+xsd). To check if the file in the folder is an actual tanach-\/xml file or something else. Tanach-\/xml needs a json file with the same schema name (tanach-\/xml.\+json), to read an process the content correctly.

Folder\+: xml -\/$>$ validation file -\/$>$ json -\/$>$precessing.

\subsection*{Structure of the json file}

Xml-\/tag \+: operation

\subsection*{Operations}

Usage on how to handle the X\+ML tags.


\begin{DoxyItemize}
\item ignore -\/ ignores the tag.
\item important
\item read -\/ simply reads the tag and displays the content/children in the X\+M\+L\+Reader.
\item desc -\/ reads the content and displays it in the File Description.
\item encoding -\/ important for the file encoding.
\item abbrev -\/ is used for the book abbreviation (searchable).
\item chapter -\/ displays the whole chapter.
\item word -\/ the simple word to display.
\item verse -\/ displays the whole verse.
\item value -\/ a numerical value (usally an attribute).
\item attrib -\/ a simple attribute (usally text).
\item link -\/ U\+RL link.
\item folder -\/ all the necessary books are in a folder.
\item single -\/ all the necessary books are in the same file. 
\end{DoxyItemize}